\chapter*{Glossario}
\nocite{*}
\section*{B}
\paragraph{Build: } trasformazione del codice sorgente in software eseguibile (tramite compilazione e \textit{linking}).

\section*{E}
\paragraph{Embedded:} vedi \textbf{Sistemi embedded}

 \section*{F}
\paragraph*{Firmware:}  programma, inteso come sequenza di istruzioni, integrato direttamente in un componente elettronico nel senso più vasto del termine (integrati, schede elettroniche, periferiche). Lo scopo del programma è quello di avviare il componente stesso e consentirgli di interagire con altri componenti tramite l'implementazione di protocolli di comunicazione o interfacce di programmazione.

\section*{K}
\paragraph*{Kanban: } termine giapponese che letteralmente significa "insegna", indica un elemento del sistema \textit{Just in time} di reintegrazione delle scorte o dei compiti svolti da un singolo individuo, che mano a mano vengono consumate.

\section*{O}
\paragraph*{Open source:} indica che il codice sorgente di un prodotto software è reso accessibile da una licenza, essa permette a terzi il diritto di poter studiare, cambiare e ridistribuire il software modificato.

\section*{S}
\paragraph*{Sistemi embedded:} tutti quei sistemi elettronici di elaborazione digitale a microprocessore progettati appositamente per una determinata applicazione ovvero non riprogrammabili dall'utente per altri scopi, spesso con una piattaforma hardware ad hoc, integrati nel sistema che controllano ed in grado di gestirne tutte o parte delle funzionalità richieste.

